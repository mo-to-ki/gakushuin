\documentclass[dvipdfmx]{report} % 文章の形式を設定
\usepackage[margin=2.5cm]{geometry} % 書式の空白を設定
\usepackage[utf8]{inputenc} % 文字コードをUTF-8に設定
\usepackage{hyperref} % 目次にリンクを付けるため
\usepackage{lipsum} % ダミーテキスト用
\usepackage{tcolorbox} % 枠を利用するため
\usepackage{amsmath} % 数式の記述を行うため
\usepackage{bm} % ベクトルを太字で表示するため
\usepackage{graphicx} % 画像を表示するため
\usepackage{float} % 画像正しい位置で表示するため
\usepackage{tensor} % テンソルを記載するため
\usepackage{multicol} % 複数段落を作成するため
\usepackage{tikz} % 図を作成するため
\usepackage{amssymb} % 特殊文字を表示するため


\title{球対称なブラックホールにおける薄い降着円盤の光学的出現}
\author{大豆生田 幹}
\date{}

\begin{document}

\maketitle % タイトルの作成
\tableofcontents % 目次の作成

% =================================
% =================================
% =================================
% =================================
% =================================
% 1章
% =================================
% =================================
% =================================
% =================================
% =================================
\chapter{チャプター}

\section{セクション}

/////////////////////////////////////////////////////////////////////////\\
/////////////////////////////////////////////////////////////////////////
\begin{equation*}
\begin{split}
	\bar{w}_j = \left( \right) \int^{}_{}
\end{split}
\end{equation*}

\begin{tcolorbox}[title=メモ用]
\begin{eqnarray*}
	1 = 0
\end{eqnarray*}
\end{tcolorbox}

\begin{equation}
\left\{ \,
\begin{aligned}
	O(\epsilon^{-2}) &: \left( \nabla_i S \right) \left( \nabla^i S \right) = 0\\
	O(\epsilon^{-1}) &: 2 \left( \nabla_i S \right) \left( \nabla^i C \right) + C \left( \square S \right) = 0\\
	O(0) &: \square C = 0
\end{aligned}
\right.
\end{equation}

\end{document}