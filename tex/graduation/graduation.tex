\documentclass[dvipdfmx]{report} % 文章の形式を設定
\usepackage[margin=2.5cm]{geometry} % 書式の空白を設定
\usepackage[utf8]{inputenc} % 文字コードをUTF-8に設定
\usepackage{hyperref} % 目次にリンクを付けるため
\usepackage{lipsum} % ダミーテキスト用
\usepackage{tcolorbox} % 枠を利用するため
\usepackage{amsmath} % 数式の記述を行うため
\usepackage{bm} % ベクトルを太字で表示するため
\usepackage{graphicx} % 画像を表示するため
\usepackage{float} % 画像正しい位置で表示するため
\usepackage{tensor} % テンソルを記載するため
\usepackage{multicol} % 複数段落を作成するため
\usepackage{tikz} % 図を作成するため
\usepackage{enumerate} % リストを作成するため
\usepackage{amssymb} % 特殊文字を表示するため


\title{超大質量天体に落下する星の光学的出現}
\author{大豆生田 幹}
\date{}

\begin{document}

\maketitle % タイトルの作成
\tableofcontents % 目次の作成
\fontsize{11pt}{11pt}\selectfont % 文字サイズの指定

% =================================
% =================================
% =================================
% =================================
% =================================
% 一般相対論
% =================================
% =================================
% =================================
% =================================
% =================================
\chapter{一般相対論}

\section{概念}

\section{ベクトル}
時空は局所的に平坦とみなすことができた。つまり、時空は局所的にユークリッド空間と同相な空間と考えることができるので、そこでは滑らかな座標が存在する。そこで、座標を以下のようにとることにする。
\[ (x^0, x^1, x^2, x^3) \]
時空上で実数値をもつ滑らかな関数$f(x^i)$と、実数$t$をパラメータとする曲線$C(t)$を考える。ここで、曲線$C(t)$上における関数$f(x^i)$の$t$微分は
\[ \frac{df}{dt} = \frac{\partial f}{\partial x^i} \frac{dx^i}{dt} \]
とかける。この右辺の成分それぞれに着目する。
座標変換$(x^0, x^1, x^2, x^3) \rightarrow (\bar{x}^0, \bar{x}^1, \bar{x}^2, \bar{x}^3)$を考えると

\begin{enumerate}[(1)\,]

\item{}
\[ v^i = \frac{dx^i}{dt}\]
では、
\begin{equation*}
\begin{split}
	\bar{v}^i &= \frac{d\bar{x}^i}{dt} = \frac{\partial \bar{x}^i}{\partial x^j} v^j
\end{split}
\end{equation*}
このような関係を満たすものを反変ベクトルと言い、上記のようにベクトルの添え字を上に書く。

\item{}
\[ w_i = \frac{df}{dx^i} \]
では、
\begin{equation*}
\begin{split}
	\bar{w}_j = \frac{df}{d\bar{x}^i}  = \frac{\partial x^i}{\partial \bar{x}^j} w_i
\end{split}
\end{equation*}
このような関係を満たすものを共変ベクトルと言い、上記のようにベクトルの添え字を下に書く。

\end{enumerate}
以上のをまとめると、ベクトルは2種類に分類できて、それぞれの座標変換は以下のように書ける。
\begin{tcolorbox}[title=反変ベクトルの変換則]
	\[ \bar{v}^i = \frac{\partial \bar{x}^i}{\partial x^j} v^j \]
\end{tcolorbox}
\begin{tcolorbox}[title=共変ベクトルの変換則]
	\[ \bar{w}_i = \frac{\partial x_j}{\partial \bar{x}^i} w_j \]
\end{tcolorbox}


\section{内積}


\section{計量}


\section{並行移動と共変微分}


\section{測地線}


\section{アインシュタイン方程式}


\section{シュバルツシルトの外部解}


% =================================
% =================================
% =================================
% =================================
% =================================
% シュバルツシルト時空における軌道
% =================================
% =================================
% =================================
% =================================
% =================================
\chapter{シュバルツシルト時空における天体の軌道}
\section{軌道の安定性}


/////////////////////////////////////////////////////////////////////////\\
/////////////////////////////////////////////////////////////////////////
\begin{equation*}
\begin{split}
	\bar{w}_j = \left( \right) \int^{}_{}
\end{split}
\end{equation*}

\begin{tcolorbox}[title=メモ用]
\begin{eqnarray*}
	1 = 0
\end{eqnarray*}
\end{tcolorbox}

\begin{equation}
\left\{ \,
\begin{aligned}
	1 &= 0\\
	1 &= 0\\
\end{aligned}
\right.
\end{equation}

\end{document}